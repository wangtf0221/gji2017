Dear Reviewer,

Thank you for being the reviewer of our manuscript. 

This manuscript presents an study on reflection inversion to update the low and
intermediate wavenumber in the deeper part of elastic model, which can provide good
initial models for elastic full waveform inversion (EFWI). 
As we know, the elastic reflection waveform inversion (ERWI)
can mitigate the nonlinearity to some extent, it is
still stuck with the cycle-skipping problem due to the objective function of waveform
fitting. 

Building initial P and S wave velocity models for EFWI through elastic
wave-equation reflections traveltime inversion (ERTI) would be effective and robust
since traveltime information relates to the background model more linearly. The wave
mode decomposition, both on the recording surface and in the underground space, are
important for ERTI. On the one hand, P/S separation of multicomponent seismograms on
the surface provides individual P or S wave data residuals. Thus, we implement the
ERTI using the L<sub>2</sub> norm of the isolated P or S wave traveltime residuals
extracted by the dynamic image warping (DIW) as objective function. On the other hand,
the underground spatial wave mode decomposition provides separated wavefields to
precondition the kernels or gradients. However, the reflection kernels in elastic
media are complicated and difficult to use, especially when calculating the gradient
of S-wave velocity. The investigation of reflection kernels show that mode
decomposition can suppress the artifacts in gradient calculation. Accordingly, a
two-step inversion strategy is adopted to effectively reduce the nonlinearity of
inversion, in which PP reflections are first used  to invert P wave velocity,
followed by S wave velocity inversion with PS reflections based on the well recovered
P wave velocity. Numerical example of Sigsbee2A model validates the effectiveness of
the algorithms and strategies for ERTI.

That is the summary of our manuscript. Hope you can give us your valuable comments on
our work. 

Best regards.

